%% The first command in your LaTeX source must be the \documentclass command.
%%
%% Options:
%% twocolumn : Two column layout.
%% hf: enable header and footer.
\documentclass[
twocolumn,
% hf,
]{ceurart}

%%
%% One can fix some overfulls
\sloppy

%%
%% Minted listings support 
%% Need pygment <http://pygments.org/> <http://pypi.python.org/pypi/Pygments>
%% \usepackage{minted}
\graphicspath{ {./figures/} }
%% auto break lines
%% \setminted{breaklines=true}

%%
%% end of the preamble, start of the body of the document source.
\begin{document}

%%
%% Rights management information.
%% CC-BY is default license.
\copyrightyear{2021}
\copyrightclause{Copyright for this paper by its authors.
  Use permitted under Creative Commons License Attribution 4.0
  International (CC BY 4.0).}

%%
%% This command is for the conference information
\conference{CIKM'21: Complex Data Challenges for Earth Observation,
  November 01--05, 2021, Online}

%%
%% The "title" command
\title{Uncertainty-aware graph-based multimodal remote sensing classification}
%

%% The "author" command and its associated commands are used to define
%% the authors and their affiliations.
\author[1]{Iain~Rolland}[orcid=0000-0002-4137-5605, email=imr27@cam.ac.uk,]

\address[1]{Department of Engineering, University of Cambridge, Cambridge, CB2 1PZ United Kingdom}
\address[2]{Joint Institute for Nuclear Research, 6 Joliot-Curie, Dubna, Moscow region, 141980, Russian Federation}

\author[2]{A Nother}[%
orcid=0000-0001-7116-9338,
email=another@vu.nl,
]
\address[3]{Vrije Universiteit Amsterdam, De Boelelaan 1105, 1081 HV Amsterdam, The Netherlands}

\author[3]{A Nother}[%
orcid=0000-0002-9421-8566,
email=another@acm.org,
]

%%
%% The abstract is a short summary of the work to be presented in the
%% article.
\begin{abstract}
  A clear and well-documented document is presented as an
  article formatted for publication by CEUR-WS in a conference
  proceedings. Based on the ``ceurart'' document class, this article
  presents and explains many of the common variations, as well as many
  of the formatting elements an author may use in the preparation of
  the documentation of their work.
\end{abstract}

%%
%% Keywords. The author(s) should pick words that accurately describe
%% the work being presented. Separate the keywords with commas.
\begin{keywords}
  paper template \sep
  paper formatting \sep
  CEUR-WS
\end{keywords}

%%
%% This command processes the author and affiliation and title
%% information and builds the first part of the formatted document.
\maketitle

\section{Introduction}

\section{Modifications}

\section{Template parameters}

\section{Front matter}

\subsection{Title Information}

\subsection{Title variants}

\subsection{Authors and Affiliations}

\subsection{Abstract and Keywords}

Abstract here.

\section{Sectioning Commands}

\section{Math Equations}

\subsection{Inline (In-text) Equations}

\subsection{Display Equations}

\section{Citations and Bibliographies}

\subsection{Some examples}

\section{Acknowledgments}

%% The acknowledgments section is defined using the "acknowledgments" environment
%% (and NOT an unnumbered section). This ensures the proper
%% identification of the section in the article metadata, and the
%% consistent spelling of the heading.
\begin{acknowledgments}
  Thanks to the developers of ACM consolidated LaTeX styles
\end{acknowledgments}

%%
%% Define the bibliography file to be used
\bibliography{sample-ceur}




\end{document}

%%
%% End of file
